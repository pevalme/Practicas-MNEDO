\documentclass[nochap]{apuntes}
\title{Métodos numéricos para EDO}
\author{Pedro Valero}
\date{13-10-2015}

% Paquetes adicionales
\usepackage{tikztools}
\usepackage{fastbuild}
\usepackage{booktabs}
\usetikzlibrary{arrows}

\begin{document}
%\maketitle
\pagestyle{plain}

%%%%%%%%%%%%%%%%%%%%%%%%%%%%%%%%%%%%%%%%%%%%%%%%%%%%%%%%%%
%%
%%         Práctica 1
%%
%%%%%%%%%%%%%%%%%%%%%%%%%%%%%%%%%%%%%%%%%%%%%%%%%%%%%%%%%%
\section{Primer apartado}
Una vez programados y ejecutados los métodos mencionados de Adams-Moulton y de Euler con los datos iniciales dados en el enunciado podemos comprobar que la previsión para la humanidad en caso de desatarse el apocalipsis es:

\begin{center}
\begin{tabular}{lccc}    \toprule
Método    & Humanos (S)  & Zombies (Z)  & Muertos infectados (R)  \\ \midrule
Adams-Moulton & $6 \cdot 10^{-6}$ & 239,05 & 262,97\\
Euler & $2 \cdot 10^{-6}$ & 239,06 & 262,96\\\bottomrule
 \hline
\end{tabular}
\end{center}

Podemos ver que el resultado es el mismo con ambos métodos, como era de esperar. Hay una mínima diferencia en la cantidad de humanos restante pero no es significativa puesto que es del orden de $10^{-6}$.

El problema que estamos considerando no permite la generación de ningún tipo de criatura de la nada. Por tanto, el número total de individuos se mantiene constante, como puede observarse en los resultados obtenidos.

La suma de individuos que obtenemos no es exactamente 502 (como son los individuos originales) pero dista muy poco y esta diferencia se debe a los errores de redondeo introducidos en cada iteración.

Veamos qué ocurre a nivel teórico:
\begin{itemize}
\item \textbf{Método de Euler}
\[y_{n+1}=y_n+hf(x_n,y_n) = \left(S-hβSZ,Z+h(βSZ+λR-αSZ),R+h(αSZ-λR)\right)\]

Sumando los términos de $y_{n+1}$ es:
\[S-hβSZ+Z+h(βSZ+λR-αSZ)+R+h(αSZ-λR) = S+Z+R\]
que es la suma de los términos en $y_n$.

Con lo que podemos ver que el método de Euler respetará (en teoría) la variación de $R+Z+S$ de forma exacta.

\item \textbf{Método de Adams-Moulton}
Definimos la función $g(y)=y_1+y_2+y_3)$.

Es obvio que $g(y_1)=g(y_0)$ pues sólo estamos imitando un paso del método de Euler.

Es sencillo ver que $g(f(x_n,y_n))=0$.

Supongamos ahora que $\forall n < N g(y_n)=g(y_0)$. Veamos qué pasa con $y_{N}$.

Lo primero que hacemos es realizar una aproximación de $y_{N}$, pues estamos empleando el método de Adams-Brashforth.

Podemos ver que
\[g(y_{N})=g(y_{N-1})+0\]

Tras esto empleamos el método en sí para obtener
\[g(y_N)=g(y_{N-1})+0=g(y_0)\]

Queda claro por tanto que este método también conserva siempre el valor de $R+S+Z$ a nivel teórico.

\section{Segundo apartado}

Si al finalizar el día la población de zombies se ve reducida al a mitad, el problema puede modelarse como 10 PVI diferentes (puesto que estamos simulando 10 días)

Con esta modificación obtenemos la siguiente previsión

\begin{center}
\begin{tabular}{lccc}    \toprule
Método    & Humanos (S)  & Zombies (Z)  & Muertos infectados (R)  \\ \midrule
Adams-Moulton & 24,1652 & 4.6630 & 473,4431\\
Euler & 24.3955 & 4.8633  & 473,0197\\\bottomrule
 \hline
\end{tabular}
\end{center}

lo que supone un futuro mucho más esperanzador para la raza humana.

\section{Tercer apartado}
En esta práctica estamos resolviendo el siguiente PVI:
\[\left\{ \begin{array}{lll}
y_1'& = & -0.0095y_1y_2\\
y_2'& = & 0.0095y_1y_2 + 0.0001y_3 - 0.005y_1y_2\\
y_3'& = & 0.005y_1y_2-0.0001y_3\\
y(0)& = & (500,2,0)
\end{array}\right.\]

Por tanto, la función con la que estamos trabajando es:
\[f(x,y)=(-0.0095y_1y_2,0.0095y_1y_2 + 0.0001y_3 - 0.005y_1y_2,0.005y_1y_2-0.0001y_3)\]

Es evidente que esta función es continua puesto que cada coordenada no es más que un polinomio. Veamos si es o no Lipschitz en la segunda variable.

\[\norm{f(x,y)-f(x,\tilde{y})} =\norm{\left(0.0095(y_1y_2-\tilde{y}_1\tilde{y}_2), 0.0045(y_1y_2-\tilde{y}_1\tilde{y}_2) + 0.0001(y_3-\tilde{y}_3),\atop0.005(y_1y_2-\tilde{y}_1\tilde{y}_2)-0.0001(y_3-\tilde{y}_3) \right)}=\]
\[=\sqrt{0.0095^2(y_1y_2-\tilde{y}_1\tilde{y}_2)^2 + \left(0.0045(y_1y_2-\tilde{y}_1\tilde{y}_2) + 0.0001(y_3-\tilde{y}_3) \right)^2 + \atop \left(0.005(y_1y_2-\tilde{y}_1\tilde{y}_2)-0.0001(y_3-\tilde{y}_3)\right)^2}=\]

\[=\sqrt{1.3 \cdot 10^{-4}(y_1y_2-\tilde{y}_1\tilde{y}_2)^2 + 2\cdot 10^{-8} (y_3-\tilde{y}_3)^2 + 1\cdot 10^{-7} (y_3-\tilde{y}_3)(y_1y_2-\tilde{y}_1\tilde{y}_2)} \]

Es sencillo ver que la función no es globalmente Lipschitz. Para ello basta con tomar:
\[\begin{array}{l}y_1, y_2 < 0, y_3 \in \real \\
\tilde{y}_1 = y_1 + ε, \tilde{y}_2 = y_2 + δ, \tilde{y}_3 = y_3\end{array} \implies \norm{f(x,y)-f(x,\tilde{y})} = 0.011(-y_1δ-y_2ε+εδ)\]

Si la función fuese Lipschitz tendríamos que
\[ 0.011(-y_1δ-y_2ε+εδ) \leq C \sqrt{ε^2+δ^2}\]
pero es sencillo ver que esto es imposible puesto que ε y δ serán tan pequeños como deseemos y ``constantes'' mientras que $y_1,y_2$ pueden hacerse tan grandes como queramos.

No obstante, el problema particular sobre el que estamos trabajando, parte de la necesidad de que $y_i>0$, puesto que estamos contando número de individuos, por lo que no tiene sentido un valor negativo.

Además, los valores de $y$ que tomemos no serán tres datos cualesquiera, sino que deberán mantener una relación y es que sus coordenadas siempre sumen lo mismo que las coordenadas del dato inicial.

Así, en el mismo ejemplo que nos ha dado el error en este caso, deberíamos escribir:
\[\begin{array}{l}y_1, y_2 > 0, y_3 \in \real^+ \\
\tilde{y}_1 = y_1 + ε, \tilde{y}_2 = y_2 -ε, \tilde{y}_3 = y_3\end{array} \implies \norm{f(x,y)-f(x,\tilde{y})} = 0.011(y_1ε-y_2ε+ε^2)\]
con lo que tendríamos:
\[0.011(ε^2+ε(y_1-y_2)) \leq 0.11(ε^2+ε(DI))\leq Cε = C\sqrt{ε^2+ε^2}\]
donde $DI$ es la suma de las coordenadas del dato inicial.

Veamos ahora el caso general.

Por comodidad vamos a definir:
\[\begin{array}{ll}
α_1 = 1.3 \cdot 10^{-4} &a = \tilde{y}_1 - y_1 \implies \tilde{y}_1 = y_1 + a\\
α_2 = 2\cdot 10^{-8} &b = \tilde{y}_2 - y_2    \implies \tilde{y}_2 = y_2 + b\\
α_3 = 10^{-7} & c = \tilde{y}_3-y_3 \implies \tilde{y}_3 = y_3
+ c\end{array}\implies a+b+c = 0\]

y reordenamos las variables escribiendo:
\[\norm{f(x,\tilde{y})-f(x,y)} = \sqrt{α_1(\tilde{y}_1\tilde{y}_2-y_1y_2)^2 + α_2 (\tilde{y}_3-y_3)^2 + α_3 (\tilde{y}_3-y_3)(\tilde{y}_1\tilde{y}_2-y_1y_2)}=\]
\[= \sqrt{α_1(y_1b+y_2a+ab)^2 + α_2c^2 + α_3c(y_1b+y_2a+ab)}=\]
\[=\sqrt{α_1y_1^2b^2+α_1y_2^2a^2+α_1a^2b^2+2α_1y_1y_2ab+2α_1y_1b^2a+2α_1y_2a^2b+α_2c^2+\atop α_3cy_1b+α_3cy_2a+α_3cab}\]

Puesto que $α_i > 0$ y los cuadrados también, podemos definir:
\[K=\max\{α_1y_1^2+α_1a^2,α_1y_2^2,α_2\}\]
que sabemos que es finito pues las $α_i$ son constantes y los valores de $y_1,a,b,c$ están acotados por la suma de las coordenadas del dato inicial, DI.

A partir de esto podemos escribir
\[\norm{f(x,\tilde{y})-f(x,y)} \leq \sqrt{K(a^2+b^2+c^2)+2α_1y_1y_2ab+2α_1y_1b^2a+2α_1y_2a^2b+\atop α_3cy_1b+α_3cy_2a+α_3cab}\]

Por otro lado podemos deducir
\[a+b+c = 0 \implies a+b = -c \implies (a+b)^2 = c^2 \implies a^2 +b^2+2ab = c^2 \implies 2ab < c^2 \]

Con una demostración similar podemos ver que
\[2bc < a^2 \text{ y que } 2ac < b^2\]
Por tanto, en aquellos lugares donde tengamos $ab$, $ac$ o $cb$ multiplicado por algo positivo podremos acotar obteniendo:
\[\norm{f(x,\tilde{y})-f(x,y)}\leq \sqrt{K(a^2+b^2+c^2)+α_1y_1y_2c^2+2α_1y_1b^2a+2α_1y_2a^2b+\atop α_3y_1a^2+α_3y_2b^2+α_3cab}\]

Tomando algún valor absoluto y agrupando podemos escribir:
\[\norm{f(x,\tilde{y})-f(x,y)}\leq \sqrt{K_1(a^2+b^2+c^2)c^2+2α_1y_1b^2|a|+2α_1y_2a^2|b|+α_3|c||a||b|}\]
con
\[K_1 = K+\max\{α_1y_1y_2,α_3y_1,α_3y_2\}\]

Por último viendo que entre $a$, $b$ y $c$ uno o dos de ellos han de ser negativos (no pueden serlo todos ni ninguno pues la suma no sería 0) tenemos diferentes casos:
\begin{itemize}
\item Sólo uno de ellos es negativo, digamos $c$. Entonces
\[α_3|c||a||b| = α_1ab|c| \leq α_1|c|c^2 \]
\item Dos de los valores, digamos $a$ y $b$ son negativos. Entonces
\[α_3|c||a||b|=α_3abc \leq α_3cc^2\]
\end{itemize}
En cualquier caso hemos podido acotar el término por una constante positiva $K_k$ multiplicada por uno de los errores al cuadrado.

Finalmente tenemos
\[\norm{f(x,\tilde{y})-f(x,y)} = \sqrt{K_2(a^2+b^2+c^2)}\]
con
\[K_2 = K_1 + \max\{2α_1y_1b^2,2α_1y_2a^2,K_k\}\]
\end{document}
